\subsection{Debates}
Os debates (consultas públicas) promovidos pela \gls{sal} são realizados por meio de ferramentas específicas, a depender do modelo de Debate Público - debate sobre texto ou debate sobre temas.

Estas ferramentas foram desenvolvidas e exploradas em outros produtos.

No que concerne ao Portal \ppod, em termos de codificação, a relação dos debates para com o Portal se dá de duas formas.

A primeira delas é a listagem dos debates, correntes e finalizados, no Portal, em página específica, e as chamadas para os debates na página principal.

Para esta primeira forma foi desenvolvido um novo \gls{cpt}, chamado ``\textit{arq\_debate\_post\_type}'', que servirá como arquivo histórico dos debates. Para cada novo debate que for ser iniciado, será preciso criar um novo conteúdo baseado neste \gls{cpt} com as informações do debate.

As informações referentes a este \gls{cpt} são:
\begin{description}
\item Título;
\item Descrição;
\item Imagem;
\item \textit{Status} - Estado do debate (Aberto ou Encerrado);
\item link - Endereço da página do debate;
\item período de - Quando se iniciou o debate;
\item período para - Quando o debate foi ou será finalizado;
\item assunto - Assunto do debate;
\item categoria - Categoria do debate;
\item fases - Quais fases ocorreram ou ocorrerão neste debate; e
\item resultados - Resultados do debate.
\end{description}

Os conteúdos referentes a este \gls{cpt} estarão disponíveis na página \url{http://pensando.mj.gov.br/debates}.

Futuramente este conteúdo pode ser evoluído para conter algumas estatísticas básicas de cada debate (como número de participantes), após encerrado, para apresentar estas informações na página de listagem dos debates.

\subsection{Proponha um Debate}
Um novo recurso que foi pensado/planejado é uma nova área para que a população possa participar do processo de formulação da Agenda dos Debates, permitindo a ela contribuir na priorização de temas a serem levados como Debates Públicos pela \gls{sal}.

Para este recurso será necessário criar um \gls{cpt} próprio, que se adeque ao modelo apresentado no produto desenvolvido pela consultora Mariana Lucchesi.

As ``telas'' para este novo fluxo de participação social já estão desenvolvidas estaticamente, mas é preciso criar os recursos do Wordpress para implementar as funcionalidades nelas presentes. As telas estão presentes no tema ``\textit{pensandoodireito-tema}'', e são as seguintes:
\begin{description}
\item \textit{page-comece-debate.php};
\item \textit{page-comece-novo-debate.php};
\item \textit{page-explore-e-vote.php};
\item \textit{page-proposta-comece-debate.php}; e
\item \textit{page-proposta-criada.php}.
\end{description}
