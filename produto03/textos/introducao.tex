%------
%Se deseja-se o capítulo listado no índice mas que apareça aqui sem numeração.
%--
%\chapter*[Introdução]{Introdução}
%\addcontentsline{toc}{chapter}{Introdução}
%------
\chapter{Introdução}
De acordo com dados divulgados pela \gls{anatel}, em Março de 2015 o Brasil atingiu a marca de 24,4 milhões de usuários de serviços de Banda Larga fixa%
\footnote{Pesquisa Anatel - \url{http://www.anatel.gov.br/dados/index.php?option=com_content&view=article&id=269} - Acesso em 26/05/2015}.
%
Além disso, a \gls{pnad}, realizada pelo \gls{ibge}, revelou que 57,3\% da população brasileira utilizou Celular ou Tablet para acessar a internet em 2013%
\footnote{\url{http://www.valor.com.br/brasil/4027294/ibge-mais-de-50-usam-celular-e-tablet-para-acessar-internet} - Acesso em 27/05/2015}.

Por conta deste aumento no acesso a tecnologias e à internet faz-se necessário que os organismos do Poder Público se utilizem desses meios para aumentar a capilaridade de suas iniciativas, em especial iniciativas como o Pensando o Direito e as consultas públicas realizadas pelo Ministério da Justiça.

Segundo com a pesquisa ``Futuro Digital em Foco Brasil 2015'' (\textit{Digital Future Focus Brazil 2015}) realizada pela consultoria ComScore os brasileiros são os líderes mundiais em tempo gasto nas Redes Sociais, chegando a 650 horas por mês%
\footnote{\url{http://blogs.oglobo.globo.com/nas-redes/post/brasileiros-gastam-650-horas-por-mes-em-redes-sociais-567026.html} - Acesso em 27/05/2015}. Assim, a integração das plataformas digitais públicas com as redes sociais se faz necessária, mas deve ser realizada de forma parcimoniosa.

O maior risco ao se realizar tal integração é perder o controle sobre o conteúdo produzido e seu histórico, e quando estamos falando de iniciativas públicas estatais é de fundamental importância se manter o histórico dos processos, pois é uma das ferramentas de transparência pública. Assim, as propostas contidas neste relatório levam este ponto em consideração, buscando, sempre que possível, garantir a manutenção dos conteúdos dentro das plataformas e infraestruturas próprias dos projetos.

\section{Objetivo}
O objetivo deste produto é propor formas de integrar os projetos vinculados ao Pensando o Direito às redes sociais mantendo independência e autonomia das mesmas, mas se aproveitando da capacidade de divulgação por elas permitida.