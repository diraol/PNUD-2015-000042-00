%------
%Se deseja-se o capítulo listado no índice mas que apareça aqui sem numeração.
%--
%\chapter*[Introdução]{Introdução}
%\addcontentsline{toc}{chapter}{Introdução}
%------
\chapter{Introdução}
De acordo com \citeonline{wearesocial}, em 2014 a penetração na população brasileira de usuários de internet por meio de dispositivos móveis chegou a 39\%, e número total de usuários de internet no Brasil chegou a 54\%.

Considerando o objetivo de participação social, almejando a maior população possível, e considerando tamanha inserção dos usuários de internet móvel, chegando a quase 72\% do número total de usuários de internet, parece ser caminho necessário desenvolver tecnologias que permitam à população participar por meio dos dispositivos móveis das consultas públicas realizadas pela \gls{sal}.

\section{Objetivo}
Este documento objetiva levantar e debater algumas alternativas que permitam aos usuários da plataforma Pensando O Direito acessar e participar das consultas públicas e demais recursos da mesma, aumentando assim a capilaridade das iniciativas de participação social do Ministério da Justiça.

\section{Site Responsivo}
A atual versão do site Pensando O Direito já foi construída utilizando conceitos de responsividade, que fazem com que o site se adapte ao dispositivo em utilização de acordo com o tamanho da tela do mesmo.

Esta abordagem é ótima e suficiente para sites que simplesmente entregam conteúdo aos usuários, mas para sites que se propõe a permitir uma interatividade com os usuários, nos quais o usuário irá realizar ações e adicionar conteúdos e comentários, ela não é suficiente. Isto pois o processo de interação com dispositivos móveis é diferente do processo em desktops e notebooks, nos quais se utilizam mouse (ou touchpad), teclado, e se possui uma tela maior.

Outra diferença importante é o desempenho dos dispositivos, visto que, de forma geral, os dispositivos móveis possuem uma capacidade de processamento reduzida e também limitações no tráfego de dados quando comparados aos demais dispositivos.

Assim, recomenda-se manter a responsividade do site, mas é preciso também o desenvolvimento de um aplicativo móvel para que os usuários possam ter uma boa usabilidade.