\chapter{Conclusão}
Para uma tomada de decisão com relação a qual caminho tomar, devem ser analisadas as condições de contorno do projeto visando o desenvolvimento do aplicativo e, posteriormente, sua manutenção e evolução. 

Neste sentido, é fundamental levar em conta que a \gls{sal} não possui uma equipe de desenvolvimento fixa e tem trabalhado com o modelo de Consultores temporários, o que impõe duas grandes consequências, sendo elas a equipe de tamanho reduzido e a alta rotatividade dos profissionais.

Outro ponto a ser considerado é a limitação econômica do projeto, e a decisão político-estratégica de priorizar a utilização de softwares e tecnologias livres e abertas.

Dessa forma, a aparentem melhor decisão seria escolher pelo modelo de desenvolvimento híbrido, utilizando o framework Cordova para empacotamento de APPs. Tal opção além de reduzir a necessidade de maior diversidade técnica na equipe também reduz a quantidade de trabalho para o desenvolvimento de aplicativo para diversas plataformas. Outra vantagem é que os desenvolvedores que atuam na plataforma web também podem contribuir com o desenvolvimento do APP, aproveitando melhor os poucos recursos destinados ao projeto.

Com relação ao framework de \gls{ui} que será utilizado, a decisão deve ser tomada com base numa experimentação inicial por parte da equipe que iniciará o desenvolvimento do aplicativo, para que a adaptação da equipe de desenvolvimento ao framework seja a mais rápida possível.
