%------
%Se deseja-se o capítulo listado no índice mas que apareça aqui sem numeração.
%--
%\chapter*[Introdução]{Introdução}
%\addcontentsline{toc}{chapter}{Introdução}
%------
\chapter{Introdução}
	O Projeto Pensando o Direito é uma iniciativa da \gls{sal}, e foi criado em 2007 para promover a democratização do processo de elaboração legislativa no Brasil.

	Tal democratização do processo de elaboração legislativa se dá em diversas frentes, sendo uma delas o processo de consulta pública, na qual abre-se espaço para a participação da sociedade civil, e outra o diálogo e parceria com instituições – Academia, instituições de pesquisas, ONGs, etc; e especialistas das áreas afins de cada proposta legislativa a ser elaborada.

	O processo de abertura, transparência e participação direta proporcionado pelo Projeto Pensando o Direito pode ser alocado junto a uma pequena quantidade de outras iniciativas de participação direta da população junto ao Estado, em especial no que tange a assuntos legais e legislativos.
Mesmo o Congresso Nacional brasileiro possui pouca experiência neste sentido.
Outros dois projetos institucionais que podem ser destacados são o e-Democracia~\cite{edemocracia}, da Câmara dos Deputados, o Participa.br~\cite{participabr}, iniciativa da Secretaria da Presidência da República e o Gabinete Digital~\cite{gabinetedigital}, do Governo do Rio Grande do Sul. 

\section{Objetivo}
O presente trabalho de consultoria tem por objetivo a melhoria das ferramentas e recursos do portal Pensando o Direito de forma que se facilite a participação da sociedade civil, que as participações e interações sejam mais qualificadas e que se consiga atingir um número maior de colaboradores e colaborações. Também pode ser considerado um objetivo do trabalho a ser desenvolvido a criação de um ecossistema de ferramentas livres de participação e colaboração que possa ser livremente reimplementado e reutilizado.

Por fim, espera-se que seja legado da consultoria a incorporação e solidificação de conceitos de metodologias ágeis~\cite{shore2007art}, além da consolidação de um processo de desenvolvimento, na \gls{sal}, colocando a equipe num novo patamar processual no que tange ao desenvolvimento de ferramentas e tecnologias, utilizando as técnicas mais atuais no mercado e no ecossistema de desenvolvimento de softwares livres.