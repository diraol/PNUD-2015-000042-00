\chapter{Desenvolvimento}
Conforme exposto inicialmente, este produto tem por objetivo gerar:
``avaliação e propostas de melhorias técnicas para ferramentas e aplicações de consulta pública, incluindo plugins''. 
As avaliações e propostas serão realizadas com referência às ferramentas utilizadas nas consultas públicas do anteprojeto de lei de \pdp e da regulamentação do \mc.
	
	Tais propostas de melhoria serão realizadas levando-se em conta características de desenvolvimento ágil, em especial o chamado programação extrema[]. No que tange aos métodos ágeis, serão utilizados os conceitos de “entregáveis” (deliverables), que podem ser entendidos como propostas de modificações que se bastam individualmente, e de sprints, que são trabalhos focados no desenvolvimento e entrega de um entregável, com prazo definido. Assim, permite-se ter um melhor acompanhamento e visualização da evolução das ferramentas e aplicações, mesmo por parte de stakeholders que não tenham interação com o código fonte, e também permite que estes entregáveis sejam colocados em produção num ritmo mais acelerado, já permitindo que os usuários do sistema se beneficiem deste. No que tange à programação extrema, o principal conceito a ser utilizado é o de programação pareada (pair programming), no qual dois desenvolvedores trabalham em dupla na resolução de um mesmo problema, com um dos programadores escrevendo código e o outro revisando, criticando e dando sugestões. Este método oferece grande vantagem, pois reduz a quantidade de erros durante o desenvolvimento, permite oferecer um resultado final de maior qualidade em termos de código, e reduz também o tempo de desenvolvimento, pois eventuais dúvidas são dirimidas com maior velocidade pela soma dos conhecimentos dos dois desenvolvedores.
	
	As análises levarão em consideração plataformas já existentes, como:
\begin{itemize}
\item{e-democracia}
\item{Participa.br}
\item{Gabinete Digital}
\end{itemize}
Além disso, também serão levados em consideração feedbacks fornecidos pelos usuários das ferramentas e serão realizadas consultas com agentes estratégicos da sociedade civil para melhor compreender a demanda da mesma referente ao incentivo ao debate na plataforma.
