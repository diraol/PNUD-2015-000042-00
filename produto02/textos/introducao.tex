%------
%Se deseja-se o capítulo listado no índice mas que apareça aqui sem numeração.
%--
%\chapter*[Introdução]{Introdução}
%\addcontentsline{toc}{chapter}{Introdução}
%------
\chapter{Introdução}
As manifestações ocorridas em Junho de 2013 trouxeram definitivamente para o debate público uma crise de representação política do atual sistema, concretizada principalmente pelo desagrado da população para com os serviços públicos, sejam eles transporte, educação, saúde, etc \cite{SANTANA2013}.
Conforme pontua \citeonline{svab2014}:
\begin{citacao}
A crescente adoção de mecanismos de participação pela Internet trouxe consigo esperança sobre a possibilidade de superação da dicotomia \textit{participação versus representação}, além de também significar incremento no processo de transparência e abertura do governo.
\end{citacao}

Em 2007 a \gls{sal} iniciou o projeto Pensando o Direito objetivando promover uma democratização do processo de elaboração legislativa no Brasil, mas é agora, com as consultas públicas do \mc\footnote{Consulta Pública da Regulamentação do Marco Civil da Internet - \url{http://participacao.mj.gov.br/marcocivil/} - Acessado em 08/04/2015} e de \pdp\footnote{Consulta Pública do Anteprojeto de Lei de Proteção de Dados Pessoais - \url{http://participacao.mj.gov.br/dadospessoais/} - Acessado em 08/04/2015} que esta iniciativa começa de fato a atingir um público muito mais amplo.
%TODO Público Mais amplo de quanto?

Com essa nova abrangência, identificam-se novos desafios, dentre eles ``a transparência e clareza na sistematização das discussões promovidas;'' \cite{svab2014}. Este, assim como outros, é um desafio para o qual ainda não se tem uma solução consolidada, principalmente se considerado o fato de que existem diversas metodologias de participação, e cada uma delas demanda um ou mais processos de sistematização.

\section{Objetivo}
Considerando o contexto exposto anteriormente, o presente relatório objetiva propor soluções para permitir uma sistematização mais automatizada, pública e transparente das consultas públicas em vigor coordenadas pela \gls{sal}, a saber Regulamentação do \mc~e Anteprojeto de Lei de \pdp.

Além disso, também faz parte dos objetivos do presente trabalho propor uma metodologia de organização do desenvolvimento das ferramentas de participação com a finalidade de tornar o processo mais organizado, transparente, público e que permita a colaboração de novos agentes, sejam eles funcionários da instituição, consultores externos ou mesmo integrantes da sociedade civil que desejem contribuir.