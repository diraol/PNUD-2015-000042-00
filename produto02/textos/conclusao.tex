\chapter{Conclusão}
Este documento apresenta a necessidade e a importância do processo de sistematização de constribuições realizadas em ambientes de colaboração, em especial quando se trata da relação entre sociedade civil e governos. Esta ainda é uma área completamente em aberto e que demanda muitas experimentações e estudos.

Existem diversas formas de a sociedade civil trabalhar em parceria com o governo, sendo uma delas a colaboração na construção das ferramentas de participação social. Para tanto, é fundamental que o processo de desenvolvimento de ferramentas esteja bem documentado e seja orientado a colaborações de terceiros. Assim, apresentamos na seção \ref{sec:metodologia-agil} uma proposta de Metodologia que visa contemplar colaboradores externos e pontuais.

Além disso, como parte do processo de aprendizado e experimentação em participação social, propõe-se aqui a implementação da ferramenta \textit{Hypotes.is} para ser utilizada na sistematização das contribuições realizadas na consulta pública do \mc e do \pdp. Posteriormente se faz necessária uma avaliação de como se deu a utilização desta ferramenta neste processo.

Por fim, são apresentadas também algumas alternativas de ferramentas e tecnologias que poderiam ser utilizadas em conjunto com o \textit{Hypotes.is} para automatizar o processo de avaliação e sistematização das contribuições.