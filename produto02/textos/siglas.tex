%\begin{siglas}
%  \item[ABNT] Associação Brasileira de Normas Técnicas
%  \item[abnTeX] ABsurdas Normas para TeX
%\end{siglas}

%\newacronym%
%[description={}]%
%{}{}{}%

%\hyphenation{Django}%
%\newglossaryentry{django}{%
%    name=Django,%
%    description={\'{e} um framework para desenvolvimento r\'{a}pido para web, escrito em %Python, que utiliza o padr\~{a}o MVT}%
%}%
%\newdualentry{mvc}{MVC}{\textit{Model View Controller}}{\textit{Design Pattern} muito conhecida e introduzida por \textit{Erich Gamma} em 1995}
%\newacronym{html}{HTML}{HyperText Markup Language}

\newacronym{sal}{SAL/MJ}{Secretaria de Assuntos Legislativos do Ministério da Justiça}
\newacronym{pnud}{PNUD}{Programa das Nações Unidas para o Desenvolvimento}
\newcommand{\mc}{Marco Civil da Internet}
\newcommand{\pdp}{Proteção de Dados Pessoais}

\newglossaryentry{issue}{
	name=\textit{issue},
	description={Uma tarefa a ser realizada. No escopo deste projeto deve-se buscar a formulação de tarefas minimalistas}
}
\newglossaryentry{milestone}{
	name=\textit{milestone},
	description={Conjunto de tarefas (\textit{issues}) com prazo de finalização, neste projeto está sendo utilizado como organizador dos \textit{Sprints}}
}
\newglossaryentry{sprint}{
	name=\textit{sprint},
	description={é a unidade básica de desenvolvimento, no caso deste projeto é o trabalho (a ser) realizado durante uma semana}
}
\newglossaryentry{integrador}{
	name=integrador,
	description={Papel na equipe de desenvolvimento do responsável por realizar a integração do que foi desenvolvido pela equipe ao repositório principal do projeto}
}

\newglossaryentry{label}{
	name=\textit{label},
	description={``Etiqueta'' que pode ser adicionada a uma \textit{issue} para melhor caracterização e organização das \textit{issues}}
}

\newglossaryentry{branch}{
	name=\textit{branch},
	plural=\textit{branches},
	description={é uma ramificação na árvore de desenvolvimento do projeto no sistema de versionamento git. Para mais informações ver: \url{http://git-scm.com/book/pt-br/v1/Ramificação-Branching-no-Git-O-que-é-um-Branch}}
}

\newglossaryentry{merge}{
	name=\textit{merge},
	description={é a ação de agregar propostas de modificação de diferentes \textit{branches} ou \textit{forks}}
}

\newglossaryentry{fork}{
	name=\textit{fork},
	description={é uma cópia do projeto que tem seu desenvolvimento acontecendo de forma independente da árvore de desenvolvimento do projeto original}
}

\newglossaryentry{hotfix}{
	name=\textit{hotfix},
	plural=\textit{HotFixes},
	description={é o desenvolvimento emergencial para a resolução de algum problema do sistema que está em produção. HotFixes possuem prioridade máxima e ocorrem fora do ciclo padrão de gestão do projeto}
}

\newglossaryentry{deploy}{
	name=\textit{deploy},
	description={ação de implementar um serviço, ou atualização de um serviço, para que o mesmo possa ser utilizado pelos usuários}
}

\newglossaryentry{synset}{
	name=\textit{synset},
	plural=\textit{synsets},
	description={é um conjunto de sinônimos cognitivos, ou seja, um conjunto de verbos que expressam um mesmo conceito}
}

\newglossaryentry{toolkit}{
	name=\textit{toolkit},
	plural=\textit{toolkits},
	description={é um conjunto de ferramentas para realizar diferentes tarefas dentro de um mesmo escopo de atividade ou trabalho}
}

\newdualentry{pullrequest}{PR}{\textit{pull-request}}{requisição realizada com propostas de modificação no código fonte do repositório principal do projeto}

\newdualentry{pln}{PLN}{Processamento de Linguagem Natual}{é uma subárea da inteligência artificial e da linguística que estuda os problemas da geração e compreensão automática de linguas humanas naturais}

\newdualentry{nlp}{NLP}{\textit{Natural Language Processing}}{termo em inglês para Processamento de Linguagem Natural}

\glsresetall
%\glsaddsall