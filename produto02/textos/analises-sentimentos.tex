\subsection{Análise de Sentimentos\label{subsec:analise-sentimentos}}
Para além da proposta de utilização da ferramenta \textit{Hypotes.is} como auxiliar do processo de sistematização das contribuições, exposta na Seção \ref{sec:anotacoes}, iremos agora indicar alguns recursos a serem melhor estudados e avaliados para o desenvolvimento de novas ferramentas que auxiliem no processo de sistematização das contribuições.

\subsubsection{WordNet.Br 1.0}
A \textbf{WordNet.Br 1.0}%
\footnote{WordNet.Br 1.0 - \url{http://143.107.183.175:21380/wordnetbr/} - Acessado em 15/04/2015}
é uma base de dados de verbos em português (Brasil) construída de forma alinhada com a versão 2.0 da WordNet de Princeton%
\footnote{WordNet.Pr - WordNet de Princeton, base de verbos na lingua inglesa - \url{http://wordnet.princeton.edu/}} (WordNet.Pr), base de verbos na lingua inglesa.

%TODO - Para que serve uma base de verbos?

Esta versão conta com 5860 verbos em 3713 \glspl{synset}. Um \gls{synset} é um conjunto de verbos que representam 